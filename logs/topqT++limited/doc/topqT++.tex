\documentclass[a4paper,11pt]{article}
\usepackage[a4paper,left=0.99in, right=0.99in,top=1.2in, bottom=1.2in]{geometry}

\usepackage{common-defs}
%\usepackage{cite}
\usepackage{graphicx}
\usepackage{bm,braket}
\usepackage{hyperref}

\numberwithin{equation}{section}

\newcommand{\question}[1]{{\bf Question:} #1}
\newcommand{\slashnbar}{\slashed{\bar n}}
\newcommand{\mixed}{{M}}
\newcommand{\mycite}[1]{{\footnote{\tt  #1}}}
\newcommand{\Litwo}{{\text{Li}_2}}
\newcommand{\Lithree}{{\text{Li}_3}}
\newcommand{\bfS}{\bm{S}}
\newcommand{\tbfS}{{\tilde \bfS}}
\newcommand{\bfH}{\bm{H}}
\newcommand{\bfw}{\bm{w}}
\newcommand{\bfZ}{\bm{Z}}
\newcommand{\bfgamma}{\bm{\gamma}}
\newcommand{\bfGamma}{\bm{\Gamma}}
\newcommand{\bfI}{\bm{1}}
\newcommand{\idop}{{1\hspace{-4pt} 1}}
\newcommand{\wii}[1]{{\bfw_{i\bar i}^{#1}}}
\newcommand{\kp}{k_+}
\newcommand{\km}{k_-}
\newcommand{\lp}{l_+}
\newcommand{\smallcomment}[1]{{\small \it (#1)}}
\newcommand{\Lp}{L_\perp}
\newcommand{\betaIJ}{\beta_{IJ}}
\newcommand{\vIJ}{v_{IJ}}
\newcommand{\sIJ}{s_{IJ}}


\allowdisplaybreaks

\title{\tt topqT++}
\author{
  Sebastian Sapeta \\ \\
  {\it Institute of Nuclear Physics, Polish Academy of Sciences}, \\ 
  {\it ul. Radzikowskiego 152, 31-342 Krak\'ow, Poland}
}

%-----------------------------------------------------------------------------
\begin{document}
\maketitle

%\tableofcontents

\begin{abstract}
  Theoretical basis and practical aspects of the program.
\end{abstract}


%-----------------------------------------------------------------------------
\section{HS functions}

The soft and the hard functions enter the cross section through the trace of
their product
%
\begin{equation}
  \text{Tr} \left[\mathbold{H}_{\iibar}(M, m_t, \cos\theta, \mu) 
                  \mathbold{S}_{\iibar}(M, m_t, \cos\theta, \LT, \mu)\right] = 
  \frac{3\as^2}{8 d_i} \sum_{n=0}^\infty  \sum^{n}_{m=0}
  \left(\asp\right)^n \LT^m\,
  \text{HS}^{(n,m)}_\iibar\,.
  %\label{eq:}
\end{equation}
%
The only source of $\LT$ is the soft function.  And the N$^n$LO correction to
the soft function contains the range of powers $[L_\perp^0,\ldots, L_\perp^n]$.
%
Hence, we have the following contributions
%
\begin{align}
  \text{HS}^{(0,0)} & 
  \quad \leftarrow  \quad \mathbold{H}^{(0)} \mathbold{S}^{(0,0)}\,, \\
  \text{HS}^{(1,1)} & 
  \quad \leftarrow  \quad \mathbold{H}^{(0)} \mathbold{S}^{(1,1)}\,, \\
  \text{HS}^{(1,0)} & 
  \quad \leftarrow  \quad \mathbold{H}^{(0)} \mathbold{S}^{(1,0)},\ 
	                  \mathbold{H}^{(1)} \mathbold{S}^{(0,0)}\,, \\
  \text{HS}^{(2,2)} & 
  \quad \leftarrow  \quad \mathbold{H}^{(0)} \mathbold{S}^{(2,2)}\,, \\
  \text{HS}^{(2,1)} & 
  \quad \leftarrow  \quad \mathbold{H}^{(0)} \mathbold{S}^{(2,1)},\ 
	                  \mathbold{H}^{(1)} \mathbold{S}^{(1,1)}\,, \\
  \text{HS}^{(2,0)} & 
  \quad \leftarrow  \quad \mathbold{H}^{(0)} \mathbold{S}^{(2,0)},\ 
	                  \mathbold{H}^{(1)} \mathbold{S}^{(1,0)},\
	                  \mathbold{H}^{(2)} \mathbold{S}^{(0,0)}\,.
\end{align}

There is a  corresponding function for each 
$\mathbold{H}^{(n)}$ and each $\mathbold{S}^{(n,m)}$ in the program. The
corresponding matrices are calculated numerically for a given set of agruments
and then they are multiplied and traced. This is much more efficient than
matrix multiplication of expressions.

%-----------------------------------------------------------------------------
\bibliographystyle{unsrt}
\bibliography{precision-qcd}

\end{document}
