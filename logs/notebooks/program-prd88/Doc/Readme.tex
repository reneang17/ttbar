\documentclass[letterpaper,aps,prd,,nofootinbib]{revtex4}
\usepackage{graphicx,amsmath,amssymb}

\newcommand{\als}{\alpha_s}
\newcommand{\ep}{\epsilon}
\newcommand{\nn}{\nonumber}
\newcommand{\lp}{L_{\perp}}

\newcommand{\alert}{\bf \color{red}}

\begin{document}


\title{The fixed order coefficients}

\author{Hai Tao Li, Chong Sheng Li, Ding Yu Shao,  Li Lin Yang, Hua Xing Zhu}



\begin{abstract}
     This note documents the use of the MATHEMATICA program.
     In this program, we provide the   $\Sigma_{i\bar{i}\leftarrow ab}^{(i,j)}$
     and the relevant functions.
\end{abstract}
\maketitle
%\tableofcontents

\section{The hard and soft functions}
The complete analytic results for the hard and soft functions matrices are provided in this programs. 
The perturbative expansions of the hard and soft functions are in paper.
The functions
{\bf{Hqq(i)}} and {\bf{Hgg(i)}} give the $\mathcal{O}(\alpha_s^{i})$  hard function
for quark-antiquark annihilation and gluon gluon fusion channels, respectively, where
$i=0,1$.
The functions {\bf{softqq(i)}} and {\bf{softgg(i)}} give the $\mathcal{O}(\alpha_s^{i})$ soft
functions and here $i=0,1,2$, where at NNLO level we only give the scale-dependent  soft function.
And these fixed order hard and soft functions can be obtained by following the example in {\bf{example.nb}}.
The traces of the product of hard functions and soft functions are written as
\begin{align}
       {\rm{Tr}}\left[
       \bm{H}_{i\bar{i}}(M,m_t, \cos\theta, \mu)
       \bm{S}_{i\bar{i}}(M,m_t,L_\perp, \cos\theta ,\mu)
       \right]= \frac{8 \alpha_s^2}{3 d_i} \sum_{i=1}  \left(\frac{\alpha_s}{4\pi}\right)^n
       {\rm{HS}}_{i\bar{i}}^{(n)}(M,m_t,L_\perp, \cos\theta ,\mu)\ ,
\end{align}
where
\begin{align}
       {\rm{HS}}^{(n)}_{i\bar{i}}(M,m_t,L_\perp, \cos\theta, \mu)=
       \sum_{m=0}^{2n} {\rm{HS}}^{(n,m)}_{i\bar{i}} L_\perp^{m}\ .
\end{align}
In this program, we define
\begin{align}
       {\rm{HSgg21}} = {\rm{HS}}^{(2,1)}_{gg}, \qquad \rm{similar\ for\ others,}
\end{align}
which are used in the $\Sigma_{i\bar{i}\leftarrow ab}$ functions.

\section{the $\Sigma_{i\bar{i}\leftarrow ab}$ Coefficients}

The partonic cross section at NNLO can be written as
\begin{equation}
  C_{i\bar{i}\leftarrow ab}(z_1,z_2,q_T,M,\cos\theta,m_t,\mu)=\sum_{n=0}
        \left(\frac{\alpha_s}{4\pi}\right)^n C^{(n)}_{i\bar{i}\leftarrow ab}(z_1,z_2,q_T,M,\cos\theta,m_t,\mu)
\end{equation}

The definition of $\Sigma_{i\bar{i}\leftarrow ab}$ have been defined in our paper. The NLO and NNLO results can be expressed as
 \begin{multline}
 \label{eq:nlo_qt}
          \int \frac{dz_1}{z_1} \frac{dz_2}{z_2} \delta(z-z_1 z_2)
          C^{(1)}_{i\bar{i}\leftarrow ab}(z_1,z_2,q_T,M,\cos\theta,m_t,\mu)=
          \\
          \Sigma^{(1,0)}_{i\bar{i}\leftarrow ab}
          \delta(q_T^2)
          +
          \Sigma^{(1,1)}_{i\bar{i}\leftarrow ab}
          \left[ \frac{1}{q_T^2}\right]_*^{[\mu^2]} +
          \Sigma^{(1,2)}_{i\bar{i}\leftarrow ab}
          \left[\frac{1}{q_T^2} \ln\frac{q_T^2}{\mu^2}\right]_*^{[\mu^2]}\ ,
 \end{multline}
 \begin{multline}
 \label{eq:nnlo_qt}
          \int \frac{dz_1}{z_1} \frac{dz_2}{z_2} \delta(z-z_1 z_2)
          C^{(2)}_{i\bar{i}\leftarrow ab}(z_1,z_2,q_T,M,\cos\theta,m_t,\mu) =
          (\Sigma^{(2,0)}_{i\bar{i}\leftarrow ab}-4 \zeta_3 \Sigma^{(2,3)}_{i\bar{i}\leftarrow ab})
          \delta(q_T^2)
         \\
          +
          (\Sigma^{(2,1)}_{i\bar{i}\leftarrow ab}+ 16 \zeta_3\Sigma^{(2,4)}_{i\bar{i}\leftarrow ab})
           \left[\frac{1}{q_T^2}\right]_*^{[\mu^2]} +
          \Sigma^{(2,2)}_{i\bar{i}\leftarrow ab}
           \left[ \frac{1}{q_T^2} \ln\frac{q_T^2}{\mu^2}\right]_*^{[\mu^2]}+
           \Sigma^{(2,3)}_{i\bar{i}\leftarrow ab}
           \left[\frac{1}{q_T^2} \ln^2\frac{q_T^2}{\mu^2}\right]_*^{[\mu^2]}+
            \Sigma^{(2,4)}_{i\bar{i}\leftarrow ab} \left[\frac{1}{q_T^2} \ln^3\frac{q_T^2}{\mu^2}\right]_*^{[\mu^2]}\ .
 \end{multline}
The function {\bf{Sigmaqqbar("ab",\ i,\ j)}} returns the coefficient $\Sigma_{q\bar{q}\leftarrow ab}^{(i,j)}$, where "ab" can be "qqbar", "qg", "gg" and "qqprime". And the function {\bf{Sigmagg("ab",\ i,\ j)}} returns the coefficient $\Sigma_{gg\leftarrow ab}^{(i,j)}$, where "ab" can be  "gg", "qg" and "qq".
In the Coefficients,  the two-loop splitting functions $P^{(2)}_{i\leftarrow j}(z)$ in the $\Sigma_{gg\leftarrow ab}^{(i,j)}$ are labeled as {\bf{Pij2(z)}}. And the function {\bf {pgg2(z), pqg2(z), pqg2(z), pqqV(z), pqqbV(z), pqqS(z)}} returns the corresponding expressions of two loop splitting  functions.

The kinematic variables used in this program are defined as
\begin{align}
      M^2&=(p_3+p_4)^2,
       \qquad t_1=(p_1-p_3)^2-m_t^2,
       \qquad u_1 = (p_1-p_4)^2-m_t^2, \nonumber \\
       \beta_t&=\sqrt{1-4m_t^2/M^2}, \qquad x_s=(1-\beta_t)/(1+\beta_t), \qquad \beta_{34}=i\pi+\ln(x_s).
\end{align}


\begin{thebibliography}{199}

\end{thebibliography}


\end{document}
